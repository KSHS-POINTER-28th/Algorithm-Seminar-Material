\documentclass{article}
\usepackage[utf8]{inputenc}
\usepackage{tasks}
\usepackage{kotex}
\usepackage{amsthm}
\usepackage{amssymb}
\usepackage{hyperref}
\usepackage{mathtools}
\usepackage{cleveref}
\usepackage{lipsum}
\usepackage{polynom}
\usepackage{array}
\usepackage{enumitem}
\usepackage{authblk}
\usepackage{listings}
\usepackage{caption}
\usepackage{setspace}
\usepackage{pgfplots}

\newtheorem{definition}{Definition}

\lstset{
    frame=single,
    language=C++,
    numbers=left,
}

\begin{document}
    \title{%
    POINTER 알고리즘 세미나 \\
    \large 제 1강 - 시간 복잡도 (Time complexity)와 분석 (Analysis)}
    \author{Kim Jun Hyeok}
    \affil{Kangwon Science Highschool}
    \maketitle
    \linespread{1.25}
    지난번 연습 문제의 풀이이다.
    \section{\href{https://www.acmicpc.net/problem/1436}{BOJ 1436}}
        \href{https://solved.ac}{solved.ac} 기준으로 Silver V인 문제이다.
        이 문제는 최대한 간단한 발상이 중요하다.
        $N$의 범위를 우선 보도록 하자.
\end{document}